\documentclass{article}

\usepackage{amsmath,amssymb}
\usepackage{hyperref}
%\usepackage[T1]{fontenc}
%\usepackage{lmodern}
\usepackage{xcolor}
\usepackage{listings}
\lstset{language=[90]Fortran,
  basicstyle=\ttfamily,
  keywordstyle=\color{red},
  commentstyle=\color{darkgray},
  numbers=left, numberstyle=\tiny, numbersep=5pt,
  backgroundcolor=\color[rgb]{0.9,0.9,0.9},
  frame=single,
  morecomment=[l]{!\ }% Comment only with space after !
}

\newcommand{\cds}[1]{\texttt{#1}}

\newcommand{\python}{\texttt{Python}}
\newcommand{\ractdm}{\textsf{rac2d}\textit{m}}
\newcommand{\ractd}{\textsf{rac2d}}
\newcommand{\radtran}{\texttt{rtrf}}
\newcommand{\structadj}{\texttt{structadj}}
\newcommand{\thermochem}{\texttt{thermochem}}
\newcommand{\makeimage}{\texttt{makeimage}}

\title{Documentation for \ractdm}
\author{Fujun Du\\
\url{fujun.du@gmail.com}}

\begin{document}

\maketitle
\tableofcontents

\section{Guideline}

The \ractdm{} package is designed to model a static disk structure and its thermochemistry.  It is derived from the \ractd{} package.  The letter ``m'' in the name stands for ``modularized''.  For the ease of implementation, the interface to each module is implemented with files.

Using script languages such as \python, different modules can be chained and iterated to achieve a specific goal.

\section{Description of modules}

\subsection{Radiative transfer and bulk radiation field solution: \radtran}

\subsubsection{Overall configuration}

\begin{lstlisting}
&conf_nml_rt
  conf_rt%struct_file = "struct.dat"
  conf_rt%opti_files = "sil.dat" "gra.dat"
  conf_rt%rad_source_files = "s1.dat"  "s2.dat"
  conf_rt%star_mass_Msun = 0.6D0
  conf_rt%nphoton = 1000000
  conf_rt%nmax_encounter = 1999999999
  conf_rt%escaped_photons_file = 'escaped_ph.dat'
  conf_rt%collect_photon   = .true.
  conf_rt%collect_lam_min_AA  = 1D0
  conf_rt%collect_lam_max_AA  = 1D8
  conf_rt%collect_nmu      = 3
  conf_rt%collect_ang_mins_deg = 0D0 4D0  80D0
  conf_rt%collect_ang_maxs_deg = 3D0 10D0 90D0
  conf_rt%nlen_lut         = 2048
  conf_rt%Tdust_min_K      = 1D0
  conf_rt%Tdust_max_K      = 2D3
  conf_rt%dist_to_Earth_pc = 51.0D0
/
\end{lstlisting}

\subsubsection{Specification of matter distribution}

\begin{lstlisting}
format ! Default to 1
number_of_species  number_of_data_points
r1  z1  n11  n21  n31 ...
r2  z2  n12  n22  n32 ...
...
\end{lstlisting}
Here \cds{nij} means the density of the \(i\)th dust species at the \(j\)th location.

\subsubsection{Specification of optical properties}

\begin{lstlisting}
number_of_data_points
angstrom1  Q_abs1  Q_sca1  g1
angstrom2  Q_abs2  Q_sca2  g2
...
\end{lstlisting}

\subsubsection{Specification of radiation sources}

Multiple radiation sources are allowed.  Each source can have a different type.  At present the following types are supported:
\begin{itemize}
  \item \cds{point}: point source
  \item \cds{sphere}: photons are emitted from the surface of a sphere
  \item \cds{isotropic}: photons are emitted from an isotropic background
  \item \cds{directional}: photons are emitted from a single direction
\end{itemize}
The following are example radiation source files:

\begin{lstlisting}
point
x_au  y_au  z_au
number_of_data_points_of_spectrum total_luminosity_cgs
angstrom1  F1_cgs
angstrom2  F2_cgs
...
\end{lstlisting}

\begin{lstlisting}
sphere
x_cen_au  y_cen_au  z_cen_au  r_sphere_Rsun
number_of_data_points_of_spectrum total_luminosity_cgs
angstrom1  F1_cgs
angstrom2  F2_cgs
...
\end{lstlisting}

\begin{lstlisting}
isotropic
! Empty line
number_of_data_points_of_spectrum energy_density_cgs
angstrom1  F1_cgs
angstrom2  F2_cgs
...
\end{lstlisting}

\begin{lstlisting}
directional
v_x  v_y  v_z
number_of_data_points_of_spectrum flux_density_cgs
angstrom1  F1_cgs
angstrom2  F2_cgs
...
\end{lstlisting}

\subsubsection{Output}

\begin{itemize}
  \item Temperature distribution
  \item Internal radiation field
  \item Escaped radiation field
\end{itemize}  

\subsection{Disk structure adjustment: \structadj}

\subsubsection{Overall configuration}

\begin{lstlisting}
&conf_nml_struct_adj
  conf_struct_adj%density_file = "dens.dat"
  conf_struct_adj%density_file_new = "dens_new.dat"
  conf_struct_adj%temperature_file = "temperature.dat"
/
\end{lstlisting}

\subsubsection{Output}

\begin{itemize}
  \item New density distribution
\end{itemize}  

\subsection{Thermochemical calculation: \thermochem}

Here we assume the thermochemical calculation is only done for a single cell.  To calculate for the whole disk just call the module in a loop or use the \cds{map} function in \python.

\subsubsection{Overall configuration}

\begin{lstlisting}
&conf_nml_thermochem
  conf_thermochem%network_file = "rate12.dat"
  conf_thermochem%init_file = "init.dat"
  conf_thermochem%enthalpy_file = "enthalpy.dat"
  conf_thermochem%Tgas = 20D0
  conf_thermochem%Tdust = 20D0
  conf_thermochem%sigdust_cm2 = 20D0
  conf_thermochem%G0 = 1D0
  conf_thermochem%Xray = 0D0
  conf_thermochem%zeta_cosmicray_H2 = 1.36D-17
  conf_thermochem%PAH_abundance = 1.6D-9
  conf_thermochem%Av = 10D0
  conf_thermochem%filename_CII = ""
  conf_thermochem%filename_OI = ""
  conf_thermochem%filename_NII = ""
  conf_thermochem%filename_SiII = ""
  conf_thermochem%filename_FeII = ""
  conf_thermochem%dt_first_step = 1D-8
  conf_thermochem%t_max = 1D6
  conf_thermochem%ratio_tstep = 1.1D0
  conf_thermochem%max_runtime_allowed = 60D0
  conf_thermochem%RTOL = 1D-6
  conf_thermochem%ATOL = 1D-30
/
\end{lstlisting}

\subsubsection{Output}

\begin{itemize}
  \item For a single location, output the gas temperature and chemical abundances as a function of time.
  \item Optional output
\end{itemize}  

\subsection{Make images: \makeimage}

\section{Examples}

\end{document}
